\section{Ulazni podaci}
  \subsection{Predavaonice}
  Predavaonica je mjesto gdje se drzi predavanje.
    \begin{itemize}
      \item Broj učione
      \item Broj mjesta
      \item Vrijeme (zauzeta/slobodna)
    \end{itemize}
  \subsection{Predmeti}
    \begin{itemize}
      \item Ime
      \item Tjedni broj sati
    \end{itemize}
  \subsection{Grupe}
    \begin{itemize}
      \item Broj članova
      \item Predmeti
    \end{itemize}
  \subsection{Profesori}
    \begin{itemize}
      \item Ime
      \item Prezime
      \item Grupe
      \item Vrijeme
      \item prioritet
    \end{itemize}

\section{Pristupi problemu}
    Trenutno imamo dva pristupa problemu.
    \subsection{Prioritet imaju grupe i profesori}
    Svakome profesoru je dodjeljen skup predmeta koji on drzi i neki skup slobodnih vremena. Dok se svi preedmeti ne popune stavljaju se u prazna mjesta.
    \section{Prioritet imaju grupe i ucione}
    Svakoj ucioni je dodjeljena grupa, tj. skup grupa koja one moze primit, tj. kapacitet ucione je veci ili jedna broju elemenata grupe.


Odlucilo se za genetski algoritam, tzv otocki genetski algoritam. Da bi mogli zapoceti algoritam uz ulazne podatke potrebni su nam nacin kodiranja rjesenja, funkcija dobrote, i model za selekciju najboljih jedinki. Trenutno predlozeni model je slijedeci:

Neka je n broj raspolozivih racunala. Na pocetku imam n otoka, svaki otok izvrsava jednu generaciju genetskog algoritma nakon cega vrsi selekciju najboljih (prema funckiji dobrote). Naj-otok, tj. otok sa najboljim rjesenjam, tj. pobjedcniki otok dobiva najbolja rijesenja sa ostalih otoka te se postupak ponavlja. Otok koji ucestalno odredjeni zaredom k puta gubi biva izbacivan sa turnira i broj otok se smanjuje za jedan.

dvije moguce varijante:
* brisemo jedan otok
* slucajno se generira jedan otok od te generacije

Uvjet zaustavljana:
* kad on misli da je dovoljno dobar, tj. kada neki otok pobjedi l (l prirodan broj).
* Vrijeme (tj. kad nam treba raspored)

Aplikacija se ce vrtjeti m ciklusa, gdje je m eksperimentalno odredjeni prirodni broji koji odgovara izvrsavanju vremena a min. Svakih a min bi se zapisivalo u bazu.



Funkcija dobrote:
Generiramo rasporeda koliko imamo grupa u obliku matrice. Iz skupa profesora i skupa predmeta za tu grupu dodjelimo u matricu, tj. na slucajan nacin rasporedimo profesora s predmetima. To napravimo za sve rasporede. Funckiju dobrote odredujemo tako da preklopimo sve rasporede.
